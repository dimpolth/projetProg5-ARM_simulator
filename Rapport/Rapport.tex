% !TEX TS-program = pdflatex
% !TEX encoding = UTF-8 Unicode

% This is a simple template for a LaTeX document using the "article" class.
% See "book", "report", "letter" for other types of document.

\documentclass[11pt]{article} % use larger type; default would be 10pt

\usepackage[utf8]{inputenc} % set input encoding (not needed with XeLaTeX)

%%% Examples of Article customizations
% These packages are optional, depending whether you want the features they provide.
% See the LaTeX Companion or other references for full information.

%%% PAGE DIMENSIONS
\usepackage{geometry} % to change the page dimensions
\geometry{a4paper} % or letterpaper (US) or a5paper or....
% \geometry{margin=2in} % for example, change the margins to 2 inches all round
% \geometry{landscape} % set up the page for landscape
%   read geometry.pdf for detailed page layout information

\usepackage{graphicx} % support the \includegraphics command and options

% \usepackage[parfill]{parskip} % Activate to begin paragraphs with an empty line rather than an indent

%%% PACKAGES
\usepackage{booktabs} % for much better looking tables
\usepackage{array} % for better arrays (eg matrices) in maths
\usepackage{paralist} % very flexible & customisable lists (eg. enumerate/itemize, etc.)
\usepackage{verbatim} % adds environment for commenting out blocks of text & for better verbatim
\usepackage{subfig} % make it possible to include more than one captioned figure/table in a single float
% These packages are all incorporated in the memoir class to one degree or another...

%%% HEADERS & FOOTERS
\usepackage{fancyhdr} % This should be set AFTER setting up the page geometry
\pagestyle{fancy} % options: empty , plain , fancy
\renewcommand{\headrulewidth}{1pt} % customise the layout...
\lhead{\textsf{Universit\'e Joseph Fourier}}\chead{}\rhead{\textsf{Projet PROG5 2014-2015}}
\lfoot{}\cfoot{\thepage}\rfoot{}

%%% SECTION TITLE APPEARANCE
\usepackage{sectsty}
\allsectionsfont{\sffamily\mdseries\upshape} % (See the fntguide.pdf for font help)
% (This matches ConTeXt defaults)

%%% ToC (table of contents) APPEARANCE
\usepackage[nottoc,notlof,notlot]{tocbibind} % Put the bibliography in the ToC
\usepackage[titles,subfigure]{tocloft} % Alter the style of the Table of Contents
\renewcommand{\cftsecfont}{\rmfamily\mdseries\upshape}
\renewcommand{\cftsecpagefont}{\rmfamily\mdseries\upshape} % No bold!

%%% END Article customizations

%%% The "real" document content comes below...

\title{Projet PROG5 2014-2015 \\
Simulateur ARM \\
Rapport de Projet}
\author{SOULIER Clément \\
SENECLAUZE Pierre \\
DUPLAN Maxime \\
BERNE Corentin}
%\date{} % Activate to display a given date or no date (if empty),
         % otherwise the current date is printed 

\begin{document}

\sffamily
\maketitle

\newpage
\section{Structure du code}
Nous avons développé l'ensemble du projet dans le squelette fournis, et nous n'avons rien modifié dedans.

\section{Liste des fonctionnalités}

\begin{tabular}{|l|c|r|}
  \hline
  Description & Fichier & Etat\\
  \hline
  Accès mémoire & memory.c & Réalisée et testée \\
  Gestion des instructions & arm\_instruction.c & Réalisée et testée \\
  Traitement des données & arm\_data\_processing.c & Réalisée et testée\\
  Accès à la mémoire & arm\_load\_store.c & Réalisée et testée\\
  Rupture de séquence & arm\_branch\_other.c & Réalisée et testée\\
  Gestion des interruptions et des exceptions & arm\_exception.c & Réalisée et testée \\
  \hline
\end{tabular}.

\section{Tests efféctués}
Deux fichiers de tests ont été créés en plus de ceux fournis. Ces deux fichiers permettent de traiter l'ensemble des instructions développées au cours du projet, et les résultats sont concluants. Les fichiers de tests fournis avec le sujet fonctionnent également sans problème.
\newline
Concernant le test des interruptions et des exceptions, nous avons modifié l'exemple 2 fournit.
 
\section{Journal de la progression}
 
Durant la majorit\'e du d\'eveloppement, nous avons travaill\'e sur des parties diff\'erentes. Cependant, nous avons fait \'etat de l'avancement de nos parties respectives chaque jour.

\subsection{Jour 1: Lundi 5 janvier}

\textbf{Tout le groupe :}
\begin{itemize}
  \item Rencontre avec les autres membres du groupe.
  \item Prise de connaissance du sujet
  \item Mise en place d'un Github.
  \item Réalisation des fonctions d'acc\`es \`a la m\'emoire.
  \item Compr\'ehension et début de la r\'ealisation du chargement des instructions.
\end{itemize}

\subsection{Jour 2: Mardi 6 janvier}
\textbf{Tout le groupe :}
\begin{itemize}
  \item Fin de la r\'ealisation g\'en\'erale du chargement des instructions.
  \item R\'epartition des diff\'erentes parties entre les membres du groupe :
  \begin{itemize}
    \item Maxime : data\_processing.
    \item Pierre : branch\_other.
    \item Cl\'ement \& Corentin : load\_store.
  \end{itemize}
\end{itemize}
\textbf{Maxime :}
\begin{itemize}
  \item Cr\'eation du squelette de switch des 16 op\'erations.
  \item D\'ebut du traitement des data\_processing\_shift.
\end{itemize}
\textbf{Pierre :}
\begin{itemize}
  \item D\'ebut et fin de branch\_other.
\end{itemize}
\textbf{Cl\'ement \& Corentin :}
\begin{itemize}
  \item R\'ealisation de load\_store
\end{itemize}

\subsection{Jour 3: Mercredi 7 janvier}
\textbf{Maxime :}
\begin{itemize}
  \item Fin du traitement des d\'ecalages pour les data\_processing\_shift.
  \item Mise en place des algorithmes basiques des instructions arithm\'etiques.
\end{itemize}
\textbf{Pierre :}
\begin{itemize}
  \item Am\'elioration et correction du traitement des instructions.
\end{itemize}
\textbf{Cl\'ement \& Corentin :}
\begin{itemize}
  \item R\'ealisation de load\_store multiple
\end{itemize}

\subsection{Jour 4: Jeudi 8 janvier}
\textbf{Maxime :}
\begin{itemize}
  \item Fin du traitement des data\_processing\_shift.
  \item Début et fin des data\_processing\_immediat.
\end{itemize}
\textbf{Pierre :}
\begin{itemize}
  \item R\'ealisation de MRS pour miscellaneous.
  \item Corrections sur les branchements.
\end{itemize}
\textbf{Cl\'ement \& Corentin :}
\begin{itemize}
  \item Correction des bugs sur les macros et du premier octet des instructions \`a 1.
  \item R\'ealisation de LDRH et STRH.
  \item Tests de LDR, STR, LDRB et STRB
\end{itemize}

\subsection{Jour 5: Vendredi 9  janvier}
\textbf{Maxime :}
\begin{itemize}
  \item Correction d'un bug de signe.
  \item Fin du fichier data\_processing.
\end{itemize}
\textbf{Pierre :}
\begin{itemize}
  \item Début de la mise en place des interruptions.
  \item Am\'elioration des traces.
\end{itemize}
\textbf{Cl\'ement \& Corentin :}
\begin{itemize}
  \item Modifications et corrections diverses sur load\_store.
\end{itemize}

\subsection{Jour 6: Lundi 12  janvier}
\textbf{Tout le groupe :}
\begin{itemize}
  \item Division des t\^aches restantes :
  \begin{itemize}
    \item Pierre \& Maxime : interruptions et exceptions.
    \item Cl\'ement \& Corentin : jeux de test.
  \end{itemize}
\end{itemize}
\textbf{Pierre \& Maxime :}
\begin{itemize}
  \item Début du traitement des interruptions.
\end{itemize}
\textbf{Cl\'ement \& Corentin :}
\begin{itemize}
  \item Tests avec les jeux d'essais fournis.
\end{itemize}

\subsection{Jour 7: Mardi 13  janvier}
\textbf{Pierre \& Maxime :}
\begin{itemize}
  \item Traitement des interruptions. De nombreux bugs trouvés et résolus pour certains.
\end{itemize}
\textbf{Cl\'ement \& Corentin :}
\begin{itemize}
  \item Création et utilisation d'un test de tri par s\'el\'ection. 
\end{itemize}

\subsection{Jour 8: Mercredi 14 janvier}
\textbf{Pierre \& Maxime :}
\begin{itemize}
  \item Fin du traitement des interruptions.
  \item Ecriture du rapport.
\end{itemize}
\textbf{Cl\'ement \& Corentin :}
\begin{itemize}
  \item Création et utilisation d'un second jeu de test.
  \item Résolution de bugs mineurs.
\end{itemize}
\textbf{Tout le groupe}
\begin{itemize}
  \item Prépartion de l'oral
\end{itemize}
A ce stade, l'ensemble du sujet est traité.

\end{document}
